\begin{abstract}
Data provenance strives for {\em explaining} how the computation was performed, by recording a trace of the execution. The provenance trace is useful for a wide-range of workflows to improve the dependability, security, and efficiency of software systems. 

In this paper, we present \projecttitle\footnote{For the double-blind review process, we have changed the name to \projecttitle.}, a {\tt POSIX}-compliant data provenance library for shared-memory multithreaded programs. The \projecttitle library is completely transparent and easy to use: it can be used as a replacement for the \pthreads library by a simple exchange of libraries linked, without even recompiling the application code.  


To achieve this result,  we present a parallel provenance algorithm that records control, data, and schedule dependencies using  a {\em Concurrent Provenance Graph} (CPG).  We implemented our algorithm to operate at the compiled binary code
level by leveraging a combination of OS-specific mechanisms, and recently released \intelpt ISA extensions as part of the Broadwell microarchitecture.  Our
evaluation on a multicore platform using applications from multithreaded benchmarks suites (PARSEC
and Phoenix) shows reasonable overheads for recording the data provenance trace. \projecttitle is an active open-source project and the tool is publicly available for use in a wide range of provenance workflows. 


%for use in a wide-range of other provenance workflows. 

%Furthermore, we briefly describe three on-going case-studies where \projecttitle is being used to improve the dependability, security, and efficiency of software systems. \projecttitle is an active open-source project and the tool is publicly available to the research community for further use in other workflows.


%\projecttitle is an active open-source project currently being used for three provenance workflows, and the tool is publicly available for use in a wide-range of provenance workflows. 


%Furthermore, we briefly describe three on-going case-studies where \projecttitle is being used to improve the dependability, security, and efficiency of software systems. \projecttitle is an active open-source project and the tool is publicly available to the research community for further use in other workflows. 

%\projecttitle is an open-source project, and the library is publicly available for use in a wide-range of provenance workflows. 

\end{abstract}
