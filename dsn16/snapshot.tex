\section{Snapshot Mechanism}
\label{sec:snapshot}
An additional challenge that we need to address in the library implementation is the excessive log data produced by \intelpt, especially for long running applications. Therefore, we further extended the library to support a live snapshot facility, where the user (or an application using \projecttitle) can analyze the provenance on-the-fly while the program is still running.

For the snapshot facility, the library periodically takes a cut of the CPG. A cut is {\em consistent} if, for any synchronization operation on object $S$ in the trace,  {\em acquire}($S$) operation being in the cut implies that corresponding {\em release}($S$) is also included in the cut~\cite{chandy-lamport}.  To achieve so, we make use of modeling synchronization primitives as {\em acquire} and {\em release} operations (described in $\S$\ref{sec:algorithms}). Each thread invokes the snapshot operation on the latest synchronization event ({\em acquire} or {\em release}) in the recorded trace.

The Intel PT interface for Perf supports 2 modes: Full Trace and Snapshot mode.
When full trace is enabled the kernel will overwrite data not yet processed by
userspace (indicated by advancing a pointer). This will result in gaps in the
trace, if userspace is not fast enough collecting data. In snapshot mode however
old data in this ring buffer is preserved so that an application can start
tracing around a certain event. By default the snapshot size is 4MB for
privileged users. Perf tool exposes this feature by installing a handler on
signal SIGUSR1, which triggers the start of a trace. XY takes the same signal
and will forward it to perf tool.
