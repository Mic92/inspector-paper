
%
% \begin{figure}[t]
%
%	\fbox{	
%	\begin{minipage}{0.9\linewidth}
%	%\vspace{-2mm}
%\begin{center}		
\begin{algorithm}[t]

\myfontsize
\SetLine
$\forall S, \forall i\in \{1, ...,T\} : C_S[i] \leftarrow 0$; // All sync clocks set to zero\\
\underline{{\bf  executeThread($t$)}}\\
\Begin{


	 {\tt initThread}($t$)\;
	\While{$t$ has not terminated}{
	
		{\tt startSub-computation}({\tt instruction})\;
		\Repeat{$t$ invokes synchronization primitive}{
			Execute {\tt instruction} of $t$\; % and trace memory accesses\\
			\If {({\tt instruction} is {\tt load} {\bf or} {\tt store})} {
			{\tt onMemoryAccess}({\tt instruction})\; %// For each load/store instr.
			}
			\If {({\tt instruction} is {\tt branch} } {
			{\tt onBranchAccess}({\tt instruction})\; %// For each load/store instr.
			}
		}
		%{\tt endThunk}();  // Memoize the end state of thunk\\
		$\alpha \leftarrow \alpha + 1$; // Increment sub-computation counter \\
		// Let $S$ denote invoked synchronization primitive\\
		{\tt onSynchronization}($S$)\;
		
		
		
	}
}


\caption{\bf  Data provenance algorithm}
\label{fig:ithreadsRecord}
\end{algorithm}
%	\end{center}
%\end{minipage}
%}
%\end{figure}
