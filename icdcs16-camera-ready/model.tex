\section{System Model}
\label{sec:model}

Before we formally describe the provenance algorithm ($\S$\ref{sec:algorithms}),  we first present the system model assumed by \projecttitle.% ($\S$~\ref{sec:model}).

\myparagraph{Memory consistency model} Our approach relies on the use of the
Release Consistency memory model (RC)~\cite{DSM-RC}, which requires that all shared memory accesses are done via synchronization primitives.
%. This memory model  requires writes made by one thread to become visible only to another thread after the first thread releases a synchronization object and before the second thread acquires a synchronization object. 
For our purposes, this model has the critical
benefit of allowing us to restrict inter-thread communication (i.e. shared
memory accesses) to the synchronization points. By reducing the number of 
points in an execution where inter-thread communication can occur, we avoid
having to track individual {\tt load}/{\tt store} instructions, %(given that any of these accesses could potentially be inter-thread communication), 
which would be extremely inefficient with current hardware. 

A natural consequence of the RC memory model is that the granularity of sub-computations for provenance in \projecttitle is the sequence of instructions between two \pthreads API calls instead of  individual {\tt load}/{\tt store} instructions. Note that the RC memory model is weaker
than, for example, the Sequential Consistency model (SC)~\cite{scLamport}, but
still guarantees correctness and liveness for applications that are data race
free. In fact,  the semantics provided by
\projecttitle is as restrictive as the POSIX specification~\cite{pthreads-spec}, which mandates that all accesses to shared data structures must be properly synchronized using 
\pthreads synchronization primitives. 

\myparagraph{Synchronization model} We support the full range of synchronization
primitives in the \pthreads API, including {\em mutexes}, {\em cond\_wait/cond\_signal}, {\em semaphores},  and {\em
barriers}. However, due to the weakly consistent RC memory model, our approach
does not support {\em ad-hoc synchronization mechanisms} such as user-defined spin locks. 
Although, ad-hoc synchronization mechanisms are used due to either flexibility
or performance reasons, in practice they are shown to be
error-prone introducing bugs or severe performance issues~\cite{adhoc-sync}.
%
%used by programmers via
%shared variables (also referred to as sync variables) in loop