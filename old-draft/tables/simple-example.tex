\begin{figure}[t]
\centering
\myfontsize
{
\begin{tabular}
{m{1.5cm} m{1.4cm} m{0.1cm} m{1.8cm} m{1.2cm}}
&\underline{{\bf Thread 1} ($T_1$)} & & \underline{ {\bf Thread 2}  ($T_2$)} &\\


/* \underline{$T_{1}.a$} */ & {\tt lock()}; && &\\

 {\tt read}=$\{x, y\}$  &if ($x > 0$) && &\\
 {\tt write}=$\{x, y\}$ &  x = ++y;&& &\\
 & {\tt unlock()};&& &\\
        &  &  $\searrow$ & & \\


&&  & {\tt lock()}; &  /* \underline{$T_{2}.a$ }*/\\
&&  & y = 2* x;    & {\tt read}=$\{x\}$  \\
&  &  & {\tt unlock()}; &  {\tt write}=$\{y\}$ \\

        &  &  $\swarrow$ & & \\
/* \underline{$T_{1}.b$} */ & {\tt lock()}; && &\\

 {\tt read}=$\{x, y\}$  &if ($x > 0$) && &\\
 {\tt write}=$\{x, y\}$ &  x = ++y;&& &\\
 & {\tt unlock()};&& &\\

\end{tabular}
}


\caption{ A simple example of shared-memory multithreading}
\label{fig:simple-example}
\end{figure}
