\begin{figure}[t]
\centering
\myfontsize
{
\begin{tabular}
{m{1.5cm} m{1.4cm} m{0.1cm} m{1.8cm} m{1.2cm}}
&\underline{{\bf Thread 1} ($T_1$)} & & \underline{ {\bf Thread 2}  ($T_2$)} &\\ 
 

/* \underline{$T_{1}.a$} */ & {\tt lock()}; && &\\
{\tt read}=$\{y\}$  &  x = ++y;&& &\\
{\tt write}=$\{x, y\}$  & {\tt unlock()};&& &\\
        &  &  $\searrow$ & & \\
   

&&  & {\tt lock()}; &  /* \underline{$T_{2}.a$ }*/\\
&&  & y = 2* x;    & {\tt read}=$\{x\}$  \\
&  &  & {\tt unlock()}; &  {\tt write}=$\{y\}$ \\  
%&  &  & \hspace{.3cm} $\downarrow$    &\\
%&   &  &{\tt lock()};   &  /* \underline{$T_{2}.b$ }*/\\
%&  &  &   y = 2*x + z;  & {\tt read}=$\{x, z\}$ \\
%& &  &  {\tt unlock()}; & {\tt write}=$\{y\}$ 
\end{tabular}
}


\caption{ A simple example of shared-memory multithreading}
\label{fig:simple-example}
\end{figure}
